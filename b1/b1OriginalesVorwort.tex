\documentclass[../main.tex]{subfiles}
\begin{document}

\markleft{Vorwort}
Im Wintersemester 2009/2010 las ich zum ersten Mal die "`Lineare Algebra I'"' an der Universität Konstanz.
Damals habe ich die Sätze, Definitionen etc. noch nicht durchnummeriert und es gab kein elektronisches Skript.
Es gab noch die akademische Viertelstunde und man konnte oftmals bis zu 20 Minuten überziehen.

\bigskip\noindent
Als ich die Vorlesung im Wintersemester 2013/2014
zum zweiten Mal las, war die akademische Viertelstunde bereits abgeschafft (offiziell ist sie
ausgesetzt, aber ich bezweifle, dass sie jemals wieder eingesetzt wird). Man musste mehr oder weniger
pünktlich die Vorlesung beenden.
Ich musste einige Abschnitte der Vorlesung tippen und teilweise als Präsentation vorführen, um mit dem Stoff
durchzukommen. Glücklicherweise erstellten die Hörer Thomas Schmidt und Michael Strecke darauf aufbauend
weitgehend unabhängig voneinander schöne, jedoch von mir nicht überprüfte
elektronische Mitschriften zu meiner Vorlesung.

\bigskip\noindent
Im Wintersemester 2017/2018 las ich die Vorlesung mit nur geringfügigen Änderungen zum dritten Mal.
Ich habe bei dieser Gelegenheit das Skript von Michael Strecke zu einem von mir autorisierten Skript umgebaut, um darauf verweisen zu können.

\bigskip\noindent
Im Wintersemester 2021/2022 lese ich die Vorlesung nun zum vierten Mal. Aufgrund der COVID-19-Pandemie wird es dieses Mal zusätzlich zum Skript
sogar eine Echtzeitübertragung und eine Aufzeichnung der Vorlesung geben, obwohl man zumindest zur Zeit den Hörsaal wieder eingeschränkt
betreten kann. Die Vorlesungszeit des Wintersemesters wurde mittlerweile um eine Woche verkürzt, so dass ich zusätzlich einige Teile in zusätzlichen Lernvideos erklären werde. All diese Videos werden bereitgestellt auf einer YouTube-Playlist:
\begin{center}
\url{https://youtube.com/playlist?list=PLbQ93L5pV-a_iL7cSoU9ZZrj3KH97NObn}
\end{center}

\bigskip\noindent
Ich bin für jegliche Hinweise zu Fehlern (auch Druckfehlern) und Anregungen dankbar und nehme diese gerne persönlich
oder per Email an
\begin{center}
\texttt{markus.schweighofer@uni-konstanz}
\end{center}
entgegen. Das Skript und den zugehörigen
\LaTeX-Quelltext
mache ich verfügbar unter:
\begin{center}
\url{http://www.math.uni-konstanz.de/~schweigh/}
\end{center}

\end{document}